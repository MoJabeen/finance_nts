% Mo Jabeen Template for docs 

\documentclass[11pt]{scrartcl} % Font size

%%%%%%%%%%%%%%%%%%%%%%%%%%%%%%%%%%%%%%%%%
% Wenneker Assignment
% Structure Specification File
% Version 2.0 (12/1/2019)
%
% This template originates from:
% http://www.LaTeXTemplates.com
%
% Authors:
% Vel (vel@LaTeXTemplates.com)
% Frits Wenneker
%
% License:
% CC BY-NC-SA 3.0 (http://creativecommons.org/licenses/by-nc-sa/3.0/)
% 
%%%%%%%%%%%%%%%%%%%%%%%%%%%%%%%%%%%%%%%%%

%----------------------------------------------------------------------------------------
%	PACKAGES AND OTHER DOCUMENT CONFIGURATIONS
%----------------------------------------------------------------------------------------

\usepackage{amsmath, amsfonts, amsthm} % Math packages

\usepackage{listings} % Code listings, with syntax highlighting

\usepackage[english]{babel} % English language hyphenation

\usepackage{graphicx} % Required for inserting images
\graphicspath{{Figures/}{./}} % Specifies where to look for included images (trailing slash required)

\usepackage{booktabs} % Required for better horizontal rules in tables

\numberwithin{equation}{section} % Number equations within sections (i.e. 1.1, 1.2, 2.1, 2.2 instead of 1, 2, 3, 4)
\numberwithin{figure}{section} % Number figures within sections (i.e. 1.1, 1.2, 2.1, 2.2 instead of 1, 2, 3, 4)
\numberwithin{table}{section} % Number tables within sections (i.e. 1.1, 1.2, 2.1, 2.2 instead of 1, 2, 3, 4)

\setlength\parindent{0pt} % Removes all indentation from paragraphs

\usepackage{enumitem} % Required for list customisation
\setlist{noitemsep} % No spacing between list items

\usepackage{array}
\newcolumntype{P}[1]{>{\centering\arraybackslash}p{#1}} %Allows centering of tables

\usepackage[
backend=biber,
style=ieee,
sorting=ynt
]{biblatex}

\addbibresource{refs.bib} %Imports bibliography file

%----------------------------------------------------------------------------------------
%	DOCUMENT MARGINS
%----------------------------------------------------------------------------------------

\usepackage{geometry} % Required for adjusting page dimensions and margins

\geometry{
	paper=a4paper, % Paper size, change to letterpaper for US letter size
	top=2.5cm, % Top margin
	bottom=3cm, % Bottom margin
	left=3cm, % Left margin
	right=3cm, % Right margin
	headheight=0.75cm, % Header height
	footskip=1.5cm, % Space from the bottom margin to the baseline of the footer
	headsep=0.75cm, % Space from the top margin to the baseline of the header
	%showframe, % Uncomment to show how the type block is set on the page
}

%----------------------------------------------------------------------------------------
%	FONTS
%----------------------------------------------------------------------------------------

\usepackage[utf8]{inputenc} % Required for inputting international characters
\usepackage[T1]{fontenc} % Use 8-bit encoding

\usepackage{fourier} % Use the Adobe Utopia font for the document

%----------------------------------------------------------------------------------------
%	HEADERS AND FOOTERS
%----------------------------------------------------------------------------------------

\usepackage{scrlayer-scrpage} % Required for customising headers and footers

\ohead*{} % Right header
\ihead*{} % Left header
\chead*{} % Centre header

\ofoot*{} % Right footer
\ifoot*{} % Left footer
\cfoot*{\pagemark} % Centre footer

%----------------------------------------------------------------------------------------
%	SECTION TITLES
%----------------------------------------------------------------------------------------
 % Include the file specifying the document structure and custom commands

%----------------------------------------------------------------------------------------
%	TITLE SECTION
%----------------------------------------------------------------------------------------

\title{	
	\normalfont\normalsize
	\vspace{20pt} % Whitespace
	{\huge Binance Trading}\\ % The assignment title
	\vspace{12pt} % Whitespace
	\rule{\linewidth}{2pt}\\ % Thick bottom horizontal rule
}

\author{\small Mo D Jabeen} % Your name

\date{\normalsize\today} % Today's date (\today) or a custom date

\begin{document}

\maketitle % Print the title

\section{Margin Trading}

Allows positions to be leveraged from traders staking their funds. Enables going long or short.

Important to note that while the position is open the traders assets act as collateral for the borrowed
funds. Margin calls are made if price shifts mean you dont cover the required margin.

Margin trading uses spot trading fees (0.1 maker and taker).

\section{Futures}

Trade contracts on the underlying asset, generally date driven with an expiry but perpetual futures allows
for trading very similar to the spot and to margin trading with lower fees and more liquidity.

\subsection{How do you close a contract?}

\textbf{Offset:} Open a position opposite to the current positions to net out the setup.\\
\textbf{Rollover:} Offset then get a new contract with a longer expiration date.\\
\textbf{Settlement:} Settled at the expiration date.

\subsection{What are futures price patterns?}

Futures price = spot price + premium (can be positive or negative)

\textbf{Contango:} Price of a futures contract is higher than expected spot price.\\
\textbf{Backwardation:} Price of the futures is lower than the expected spot price.

\subsection{What are the commission fees?}

\textbf{USD-M:} maker=0.02, taker=0.04\\
\textbf{Coin-M:} maker=0.01, taker=0.05 

\[ Commission\:fee = notional\:value * fee\:rate \]
\[ Notional\:Value = (Number\:of\:contracts*contract\:size)/trade\:price \]

Fee is charged at entry and exit.

\subsection{What is funding rates ?}

Periodic payment is taken from traders, which is based on the difference between contract price and the
underlying asset spot price. This fee is designed to force the convergence between spot price and 
contract.

\[ Funding\:Amount = Nominal\:value\:of\:position * funding\:rate \]
\[ Nominal\:value = Mark\:price * contract\:size \]

Funding rate is taken every 8 hours at set times: 0:00, 8:00, 16:00 UTC.

\[ Funding\:rate = average\:premium\:index + clamp(interest\:rate - premium\:index, 0.05,-0.05) \] 

If interest rate - premium index is between -0.05 and 0.05 then the funding rate is equal to the
interest rate. Interest rate 0.03\%.

\subsection{What are the different prices?}

\textbf{Last price:} Last cost of contract (used to calculate realized P/L)\\
\textbf{Mark price:} Similar to the spot price (unrealized P/L)

\subsection{What triggers a margin call ?}

Liquidation is triggered at a margin call.

\[ Collateral = Inital\:collateral + realised\:PnL + unrealised\:PnL < maintenance margin \]

\[ margin\:ratio = maintenance\:margin/margin\:balance \]

Margin ratio is suggested to be < 80\%.

If a margin call occurs your position will be liquidated and all open position closed. It may not be
completely liquidated depends on position size.

\textbf{There will also be a liquidation fee based on the amount liquidated!!}

\subsection{What happens if you go bankrupt ?}

If the wallet hits a negative amount after liquidation, the fund will set the balance back to 0.

\subsection{What is the insurance funds ?}

The insurance fund covers bankrupted traders, its is funded from funding rates.

\subsection{What is auto-deleveraging ?}

If the fund can not cover bankruptcy, counter parties will be liquidated. Most profitable and higher
leverage trades are liquidated first. There are indicators showing your position in the auto deleverage
queue.

\section{Profit and loss}

For USD based:

\[ Profit/Loss = (Entry\:Price - Exit\:Price)* position\:size \]

(if a short multiply by -1)

For coin based:

\[ Profit/Loss = (1/Entry\:Price - 1/Exit\:Price) * position\:size \]

\subsection{Open loss}

When shorting there is an open loss:

\[ Open\:loss = order\:size * Abs(min(0,-1*(mark\:price - entry\:price)))\]

\subsection{Unrealized P/L}

Uses mark price instead of exit price. 

\[ ROE = unrealised\:P/L \div \frac{position\:size * mark\:price}{leverage} \]
\[ ROE = \frac{unrealised\:P/L}{entry\:margin} \] 


\[ ROE = \frac{Profit}{Inital Margin} = (\frac{Exit}{Entry} - 1) * leverage \]

\subsection{Target Price}

\[ LONG: Target\:price = entry\:price*(1+ \frac{ROE}{leverage}) \]
\[ SHORT: Target\:Price = entry\:price * (1-\frac{ROE}{leverage}) \]

\subsection{Price Index}

This is essentially the spot price of the underlying asset. The average price on the major markets.

\subsection{Mark Price}

The primary component of the mark price is the price index. The mark price is used to calculate the
unrealised P/L and determine if a margin call is required.

\[ Mark\:price = Median*(Price1,Price2,Contract\:Price) \]
\[ Price1 = Price\:Index * (Last\:Funding\:Rate*\frac{Time until funding}{8}) \]
\[ Price2 = Price\:Index + Moving\:average_{5min}(\frac{Bid1+Ask1}{2} - Price\:Index) \]

The price is the controlled to be between the contract price, the diff between the contract price and
spot price and the funding rate at that moment.

\subsection{Difference between coin based and USD based futures}

Coin based is settled in the contracts underlying asset.\\
USD based is settled in USDT or BUSD.

\section{Other}

\section{Binance leveraged tokens}

Tokens traded on the spot market that allow access to leverage without holding future contracts.

Uses variable leverage not constant, which has benefits for longer term holding.Charges a daily fee of
0.01\% fee.

\subsection{What is Time weighted average price (TWAP) ?}

Attempts to have a minimum impact on the market with large orders. Will break it into smaller orders
executed over a period of time.

\subsection{Liquidity Analysis ?}

Binance offers liquidity analysis tools.

\subsection{Volume participation algorithm}

Trade at a pace matching a portion of the real time market volume with respect to the target.

Good to use when making orders larger than the market liquidity while trying to minimize impact on the
market.

\subsection{Binance Portfolio management program}

Asset portfolio to track equity, margin value and maintenance margin requirements.

%----------------------------------------------------------------------------------------
%	FIGURE EXAMPLE
%----------------------------------------------------------------------------------------

% \begin{figure}[h] % [h] forces the figure to be output where it is defined in the code (it suppresses floating)
% 	\centering
% 	\includegraphics[width=0.5\columnwidth]{IMAGE_NAME.jpg} % Example image
% 	\caption{European swallow.}
% \end{figure}

%----------------------------------------------------------------------------------------
% MATH EXAMPLES
%----------------------------------------------------------------------------------------

% \begin{align} 
% 	\label{eq:bayes}
% 	\begin{split}
% 		P(A|B) = \frac{P(B|A)P(A)}{P(B)}
% 	\end{split}					
% \end{align}

%----------------------------------------------------------------------------------------
%	LIST EXAMPLES
%----------------------------------------------------------------------------------------

% \begin{itemize}
% 	\item First item in a list 
% 		\begin{itemize}
% 		\item First item in a list 
% 			\begin{itemize}
% 			\item First item in a list 
% 			\item Second item in a list 
% 			\end{itemize}
% 		\item Second item in a list 
% 		\end{itemize}
% 	\item Second item in a list 
% \end{itemize}

%------------------------------------------------

% \subsection{Numbered List}

% \begin{enumerate}
% 	\item First item in a list 
% 	\item Second item in a list 
% 	\item Third item in a list
% \end{enumerate}

%----------------------------------------------------------------------------------------
%	TABLE EXAMPLE
%----------------------------------------------------------------------------------------

% \section{Interpreting a Table}

% \begin{table}[h] % [h] forces the table to be output where it is defined in the code (it suppresses floating)
% 	\centering % Centre the table
% 	\begin{tabular}{l l l}
% 		\toprule
% 		\textit{Per 50g} & \textbf{Pork} & \textbf{Soy} \\
% 		\midrule
% 		Energy & 760kJ & 538kJ\\
% 		Protein & 7.0g & 9.3g\\
% 		\bottomrule
% 	\end{tabular}
% 	\caption{Sausage nutrition.}
% \end{table}

%----------------------------------------------------------------------------------------
%	CODE LISTING EXAMPLE
%----------------------------------------------------------------------------------------

% \begin{lstlisting}[
% 	caption= Macro definition, % Caption above the listing
% 	language=python, % Use Julia functions/syntax highlighting
% 	frame=single, % Frame around the code listing
% 	showstringspaces=false, % Don't put marks in string spaces
% 	numbers=left, % Line numbers on left
% 	numberstyle=\large, % Line numbers styling
% 	]

% 	CODE

% \end{lstlisting}

%----------------------------------------------------------------------------------------
%	CODE LISTING FILE EXAMPLE
%----------------------------------------------------------------------------------------

% \lstinputlisting[
% 	caption=Luftballons Perl Script., % Caption above the listing
% 	label=lst:luftballons, % Label for referencing this listing
% 	language=Perl, % Use Perl functions/syntax highlighting
% 	frame=single, % Frame around the code listing
% 	showstringspaces=false, % Don't put marks in string spaces
% 	numbers=left, % Line numbers on left
% 	numberstyle=\tiny, % Line numbers styling
% 	]{luftballons.pl}

%------------------------------------------------

\end{document}