% Mo Jabeen Template for docs 

\documentclass[11pt]{scrartcl} % Font size

\input{structure.tex} % Include the file specifying the document structure and custom commands

%----------------------------------------------------------------------------------------
%	TITLE SECTION
%----------------------------------------------------------------------------------------

\title{	
	\normalfont\normalsize
	\vspace{20pt} % Whitespace
	{\huge Binance Trading}\\ % The assignment title
	\vspace{12pt} % Whitespace
	\rule{\linewidth}{2pt}\\ % Thick bottom horizontal rule
}

\author{\small Mo D Jabeen} % Your name

\date{\normalsize\today} % Today's date (\today) or a custom date

\begin{document}

\maketitle % Print the title

\section{Margin Trading}

Allows positions to be leveraged from traders staking their funds. Enables going long or short.

Important to note that while the position is open the traders assets act as collateral for the borrowed
funds. Margin calls are made if price shifts mean you dont cover the required margin.

Margin trading uses spot trading fees (0.1 maker and taker).

\section{Futures}

Trade contracts on the underlying asset, generally date driven with an expiry but perpetual futures allows
for trading very similar to the spot and to margin trading with lower fees and more liquidity.

\subsection{How do you close a contract?}

\textbf{Offset:} Open a position opposite to the current positions to net out the setup.\\
\textbf{Rollover:} Offset then get a new contract with a longer expiration date.\\
\textbf{Settlement:} Settled at the expiration date.

\subsection{What are futures price patterns?}

Futures price = spot price + premium (can be positive or negative)

\textbf{Contango:} Price of a futures contract is higher than expected spot price.\\
\textbf{Backwardation:} Price of the futures is lower than the expected spot price.

\subsection{What are the commission fees?}

\textbf{USD-M:} maker=0.02, taker=0.04\\
\textbf{Coin-M:} maker=0.01, taker=0.05 

\[ Commission\:fee = notional\:value * fee\:rate \]
\[ Notional\:Value = (Number\:of\:contracts*contract\:size)/trade\:price \]

Fee is charged at entry and exit.

\subsection{Subsection}

paragraph

\section{Understanding Text}

\subsubsection{subsubsection}

%----------------------------------------------------------------------------------------
%	FIGURE EXAMPLE
%----------------------------------------------------------------------------------------

% \begin{figure}[h] % [h] forces the figure to be output where it is defined in the code (it suppresses floating)
% 	\centering
% 	\includegraphics[width=0.5\columnwidth]{IMAGE_NAME.jpg} % Example image
% 	\caption{European swallow.}
% \end{figure}

%----------------------------------------------------------------------------------------
% MATH EXAMPLES
%----------------------------------------------------------------------------------------

% \begin{align} 
% 	\label{eq:bayes}
% 	\begin{split}
% 		P(A|B) = \frac{P(B|A)P(A)}{P(B)}
% 	\end{split}					
% \end{align}

%----------------------------------------------------------------------------------------
%	LIST EXAMPLES
%----------------------------------------------------------------------------------------

% \begin{itemize}
% 	\item First item in a list 
% 		\begin{itemize}
% 		\item First item in a list 
% 			\begin{itemize}
% 			\item First item in a list 
% 			\item Second item in a list 
% 			\end{itemize}
% 		\item Second item in a list 
% 		\end{itemize}
% 	\item Second item in a list 
% \end{itemize}

%------------------------------------------------

% \subsection{Numbered List}

% \begin{enumerate}
% 	\item First item in a list 
% 	\item Second item in a list 
% 	\item Third item in a list
% \end{enumerate}

%----------------------------------------------------------------------------------------
%	TABLE EXAMPLE
%----------------------------------------------------------------------------------------

% \section{Interpreting a Table}

% \begin{table}[h] % [h] forces the table to be output where it is defined in the code (it suppresses floating)
% 	\centering % Centre the table
% 	\begin{tabular}{l l l}
% 		\toprule
% 		\textit{Per 50g} & \textbf{Pork} & \textbf{Soy} \\
% 		\midrule
% 		Energy & 760kJ & 538kJ\\
% 		Protein & 7.0g & 9.3g\\
% 		\bottomrule
% 	\end{tabular}
% 	\caption{Sausage nutrition.}
% \end{table}

%----------------------------------------------------------------------------------------
%	CODE LISTING EXAMPLE
%----------------------------------------------------------------------------------------

% \begin{lstlisting}[
% 	caption= Macro definition, % Caption above the listing
% 	language=python, % Use Julia functions/syntax highlighting
% 	frame=single, % Frame around the code listing
% 	showstringspaces=false, % Don't put marks in string spaces
% 	numbers=left, % Line numbers on left
% 	numberstyle=\large, % Line numbers styling
% 	]

% 	CODE

% \end{lstlisting}

%----------------------------------------------------------------------------------------
%	CODE LISTING FILE EXAMPLE
%----------------------------------------------------------------------------------------

% \lstinputlisting[
% 	caption=Luftballons Perl Script., % Caption above the listing
% 	label=lst:luftballons, % Label for referencing this listing
% 	language=Perl, % Use Perl functions/syntax highlighting
% 	frame=single, % Frame around the code listing
% 	showstringspaces=false, % Don't put marks in string spaces
% 	numbers=left, % Line numbers on left
% 	numberstyle=\tiny, % Line numbers styling
% 	]{luftballons.pl}

%------------------------------------------------

\end{document}