% Mo Jabeen Template for docs 

\documentclass[11pt]{scrartcl} % Font size

\input{structure.tex} % Include the file specifying the document structure and custom commands

%----------------------------------------------------------------------------------------
%	TITLE SECTION
%----------------------------------------------------------------------------------------


\title{	
	\normalfont\normalsize
	\vspace{20pt} % Whitespace
	{\huge Quant}\\ % The meh
	\vspace{12pt} % Whitespace
	\rule{\linewidth}{2pt}\\ % Thick bottom horizontal rule
}

\author{\small Dainish Jabeen} % Your name

\date{\normalsize\today} % Today's date (\today) or a custom date

\begin{document}

\maketitle % Print the title

\section{Maths used in Quantitative Finance?}

The main areas are differential equations and probability theory.\\

Useful mathematical tools:

\begin{itemize}
	\item Asymptotic analysis
	\item Series solution
	\item Discretisation methods
	\item Green functions
\end{itemize}

\section{Risk}

Simplest method is using the std of portfolio returns, high standard deviation shows high risk. 
The central limit theory suggest the average of returns will be normally distributed allowing 
the use of confidence interval.\\

\textbf{Uncertainty} is when the probabilities of risk are unknown. 
Risk is based on known probabilities.\\

\textbf{Operational Risk:} Company and technology infrastructure risk.

\subsection{Kelly Criterion}

Most funds are trying to balance the growth rate via leverage with the risk via drawdown. 
A leverage balancing tool is the Kelly Criterion.

\[ f_i = \frac{\mu_i}{\sigma_i^2} \]

\begin{itemize}
	\item i:Strategy
	\item f:Vector of allocation of length N
	\item $\mu$ Average excess return
	\item $\sigma$ Variation of excess return
\end{itemize}

Should be updated periodically using a trailing period over a window (3-6 months). The value
can be considered as an upper bound for the leverage used.

\subsubsection{Assumptions}

\begin{itemize}
	\item Gaussian return
	\item Focus on profit
	\item All profits are reinvested
	\item All strategies are independent
\end{itemize}

\subsection{Value at Risk (VaR)}

VaR is the amount that could be lost from a position, portfolio etc. Generally the maximum loss for a chosen
confidence interval over a time period.\\

\[ P(L\leq VaR) = 1 -c \]

\begin{itemize}
	\item c: confidence level
\end{itemize}

The method used is simulate many realisations of the portfolio using monte carlo method. Using the distribution 
of values to define a confidence interval. Or can just use a historical data for the distribution. However,
assumes normal market activity.\\

Time period for VaR used should have a minimum market impact; if applicable.

\subsubsection{Methods to calculate}

\begin{itemize}
	\item Variance Covariance (Normal Dist)
	\item Monte Carlo (Non Normal Dist)
	\item Historic boot strapping
\end{itemize}

\subsection{Crash Metrics}

Portfolio assessment in extreme market conditions.

\subsection{Properties of coherent risk measurement}

\subsubsection{Sub-additivity}

\[ \rho (X+Y) \leq \rho (X) + \rho (Y) \]

The risk of two portfolios combined cant be worse than the individual risks added.

\subsubsection{Monotonicity}

\[ if X\leq Y \; then \; \rho (X) \leq \rho(Y) \]

\subsubsection{Positive Homogeneity}

\[ \lambda >0 ; \rho(\lambda X) \leq \lambda \rho (X) \]

\subsubsection{Translation Invariance}

\[ constant\; c; \rho (X+c) = \rho (X) -c \]

\section{Performance Measures}

\subsection{General}

Important to compare the performance of strategies to limit opportunity cost, as capital to be allocated is
limited.\\

General measures:
\begin{itemize}
	\item Return
	\item Drawdown
	\item Risk
	\item Risk/Reward ratio
\end{itemize}

\subsubsection{Returns}

\[ Total\;Return = \frac{P_f - P_i}{P_i} *100 \]

\begin{itemize}
	\item $P_f$: Final portfolio value
	\item $P_i$: Inital portfolio value
\end{itemize}

Portfolio value over time is an equity curve.

\subsubsection{Drawdown}

Drops from high water marks, this being the top of an equity curve.\\

\textbf{Maximum Drawdown:} Largest percentage drop from previous peak to current/previous trough.\\

\textbf{Drawdown duration:} Period of this max drawdown event.\\

Can drawdown plot this over a time series.

\subsection{Trade Analysis}

\begin{itemize}
	\item Number of wins and losses
	\item Mean profit
	\item Win/Loss Ratio
\end{itemize}

\textbf{Trend Following trades:} Uses found trends to make large trades to overcome small losses.\\

\textbf{Pair trading mean reverting:} Many small wins that can be wiped by big loss.

\subsection{Risk Adjusted}

Usually adjusted for risk, as the high return with low risk is better than high return with high risk.\\

Most popular measurement is the sharpe ratio.

\subsubsection{Sharpe ratio}

\[ Sharpe=\frac{\mu - r}{\sigma } \]

\begin{itemize}
	\item $\mu$: Return over period (Must use profits)
	\item $r$: Risk free rate over period (I.e Bond interest rate)
	\item $\sigma $: std of returns 
\end{itemize}

Poor at characterising tail risk, as based on normal dist.\\

Generally S>1 is bad.

\subsubsection{Annualised Sharpe}

\[ S_A = \sqrt{N} * Sharpe \]

\begin{itemize}
	\item N: Periods (used in Sharpe) over a year 
\end{itemize}

\subsubsection{Sortino Ratio}

Same as sharpe but uses semi-variance which only has down side risk of to calculate the std.

\subsubsection{Treynor Ratio}

\[ Treynor=\frac{\mu -r}{\beta } \]

\begin{itemize}
	\item $\beta $: Systematic risk
\end{itemize}

\subsubsection{Information ratio}

\[ Info = \frac{\mu -r}{Tracking\;error} \]

\subsubsection{Calmar}

\[ CALMAR = \frac{E(R_a-R_b)}{max\;drawdown}\]

\section{Volatility}

Generally measured using annualised std of the returns. But this is not true volatility is
in truth an instantaneous measurement of randomness not historic.

\section{Back testing}

\subsection{Bias}

There a few types of bias that can cause inaccuracy in back testing:

\subsubsection{Optimisation bias}

Overfitting of model parameters via back testing, can reduce impact via sensitivity analysis.

\subsubsection{Look ahead bias}

\begin{itemize}
	\item Using data which at the time during back testing should not be available. 
	\item Data used to train is then used to test.
	\item Using aggregate values ie average/max/min should be period lagged, 
	as wont be available at the most recent time point.
\end{itemize}

\subsubsection{Survivorship bias}

Using only assets that have survived through a period of time. Avoid using data sets with stock delisiting.

\subsubsection{Cognotive bias}

Awareness of the psychological difficulty you can endure a large period of drawdown.

\subsection{Exchange Issues}

\subsubsection{Order type}

Market or limit.

\subsubsection{Price Consolidation}

Careful of composite feeds combining data from exchanges. As a single exchange will be used
to execute the trade.\\

Foreign exchanges may not require the storage of the trade size and price.

\subsubsection{Transaction Cost}

\begin{itemize}
	\item Commission
	\item Stamp Duty (UK Shares ~ 0.5\%)
	\item Slippage
	\item Market Impact (For larger orders on illiquid markets)
	\item Spread (bid - ask price)
\end{itemize}

%----------------------------------------------------------------------------------------
%	FIGURE EXAMPLE
%----------------------------------------------------------------------------------------

% \begin{figure}[h] % [h] forces the figure to be output where it is defined in the code (it suppresses floating)
% 	\centering
% 	\includegraphics[width=0.5\columnwidth]{IMAGE_NAME.jpg} % Example image
% 	\caption{European swallow.}
% \end{figure}

%----------------------------------------------------------------------------------------
% MATH EXAMPLES
%----------------------------------------------------------------------------------------

% \begin{align} 
% 	\label{eq:bayes}
% 	\begin{split}
% 		P(A|B) = \frac{P(B|A)P(A)}{P(B)}
% 	\end{split}					
% \end{align}

%----------------------------------------------------------------------------------------
%	LIST EXAMPLES
%----------------------------------------------------------------------------------------

% \begin{itemize}
% 	\item First item in a list 
% 		\begin{itemize}
% 		\item First item in a list 
% 			\begin{itemize}
% 			\item First item in a list 
% 			\item Second item in a list 
% 			\end{itemize}
% 		\item Second item in a list 
% 		\end{itemize}
% 	\item Second item in a list 
% \end{itemize}

%------------------------------------------------

% \subsection{Numbered List}

% \begin{enumerate}
% 	\item First item in a list 
% 	\item Second item in a list 
% 	\item Third item in a list
% \end{enumerate}

%----------------------------------------------------------------------------------------
%	TABLE EXAMPLE
%----------------------------------------------------------------------------------------

% \section{Interpreting a Table}

% \begin{table}[h] % [h] forces the table to be output where it is defined in the code (it suppresses floating)
% 	\centering % Centre the table
% 	\begin{tabular}{l l l}
% 		\toprule
% 		\textit{Per 50g} & \textbf{Pork} & \textbf{Soy} \\
% 		\midrule
% 		Energy & 760kJ & 538kJ\\
% 		Protein & 7.0g & 9.3g\\
% 		\bottomrule
% 	\end{tabular}
% 	\caption{Sausage nutrition.}
% \end{table}

%----------------------------------------------------------------------------------------
%	CODE LISTING EXAMPLE
%----------------------------------------------------------------------------------------

% \begin{lstlisting}[
% 	caption= Macro definition, % Caption above the listing
% 	language=python, % Use Julia functions/syntax highlighting
% 	frame=single, % Frame around the code listing
% 	showstringspaces=false, % Don't put marks in string spaces
% 	numbers=left, % Line numbers on left
% 	numberstyle=\large, % Line numbers styling
% 	]

% 	CODE

% \end{lstlisting}

%----------------------------------------------------------------------------------------
%	CODE LISTING FILE EXAMPLE
%----------------------------------------------------------------------------------------

% \lstinputlisting[
% 	caption=Luftballons Perl Script., % Caption above the listing
% 	label=lst:luftballons, % Label for referencing this listing
% 	language=Perl, % Use Perl functions/syntax highlighting
% 	frame=single, % Frame around the code listing
% 	showstringspaces=false, % Don't put marks in string spaces
% 	numbers=left, % Line numbers on left
% 	numberstyle=\tiny, % Line numbers styling
% 	]{luftballons.pl}

%----------------------------------------------------------------------------------------
%	BIB EXAMPLE
%----------------------------------------------------------------------------------------

% Using \texttt{biblatex} you can display a bibliography divided into sections, depending on citation type. 
% Let's cite! Einstein's journal paper \cite{einstein} and Dirac's book \cite{dirac} are physics-related items. 
% Next, \textit{The \LaTeX\ Companion} book \cite{latexcompanion}, Donald Knuth's website \cite{knuthwebsite}, \textit{The Comprehensive Tex Archive Network} (CTAN) \cite{ctan} are \LaTeX-related items; but the others, Donald Knuth's items, \cite{knuth-fa,knuth-acp} are dedicated to programming. 

% \medskip

% \printbibliography[
% heading=bibintoc,
% title={Whole bibliography}
% ] %Prints the entire bibliography with the title "Whole bibliography"

% %Filters bibliography
% \printbibliography[heading=subbibintoc,type=article,title={Articles only}]
% \printbibliography[type=book,title={Books only}]

% \printbibliography[keyword={physics},title={Physics-related only}]
% \printbibliography[keyword={latex},title={\LaTeX-related only}]

\end{document}